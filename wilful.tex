 \documentclass {article}
\pagenumbering {arabic}
\begin{document}
\title{AN ONLINE ORDERINGSYSTEM FOR REMA RESTAURANT}
\author {AMPAIRE WILFUL 15/U/21173   215021555}
\maketitle

\section {Introduction}

\subsection{ Research Background }
 Rema restaurant which initially started as a small restaurant serving countable number of customers a day in Makerere University but has recently gained popularity and increased the number of customers both within the university and the surrounding places. This has been due to the affordable prices on their delicious meals, a cool environment and the fact that they are suited within the university attracting a high number of not only students but also workers especially lunch hour. This has led to a rapid increase in the number of customers. However, this high number of customers has not only come with congestion of the place but also people are delayed and others end up giving up on lunch due to the long lines .They carried out research to address this problem and find possible solutions ,and then planned to develop an online ordering system called “rema food” where customers can place their orders online ,state their location on the website and the restaurants will get this notification ,reply to the customers that they will deliver with the ordered items or go to the restaurant when the orders were made already in advance.

\subsection { Problem Statement }
Due to the long queues students get caught up by time to go for their next lectures, other customers first wait for other customers to finish eating for them to get seats which is time wasting and others totally give up on going to this restaurant for lunch since it takes quite some time to get solved and this reduces the number of customers for this restaurant.   To address this problem the restaurant management sat and decided to increase the time of serving from one hour which was from (1:00-2:00)pm to two hours from (12:00-2:00)pm but unfortunately it has not worked since most students leave class at 1:00pm to go for lunch and still they would be many.

\section{ Goals and Objectives}
\subsection{ Main Objective}
To develop an online ordering system for rema restaurant to reduce on the associated problems with the long lines and service delivery.
	
\subsection {Specific Objectives}

\begin {enumerate}
\item	To observe  the number of customers that order for the services at the respective time interval.
\item	To educate their customers how to use our system that we have developed.
\item	To discover out which requirements are needed for the website.
\item	To find out the functions the system will provide.
\end {enumerate}

\section{Research Scope} 
\subsection	{Geographical scope}
The study will focus at Makerere University and the surrounding places like kikoni, kikumi kikumi, wandengaya.

\section{Research Significance}
This system will increase on the number of customers for the restaurant as it is very easy and convenient to order online.  
It will help students to reduce on the time they waste while waiting for their orders to be taken since orders will be made online.
It will increase on the sales of the restaurant as new customers will be attracted.

\end{document}